%!TEX root = ../thesis.tex

\thispagestyle{plain}

\section*{Abstract}

% BACHELOR'S THESIS ONLY: AStuPO §21 (7).4: There MUST be at least ONE abstract. It MUST be in GERMAN if the Bachelor's Thesis is not written in German.

% MASTER'S THESIS ONLY: AStuPO §21 (7).4: There MUST be at least ONE abstract. It MUST be in GERMAN OR ENGLISH if the Master's Thesis is not written in German.

Mobile devices are becoming more and more immersed in our daily lives, making them attractive targets for threats such as malware, data exfiltration, and unauthorized access and even end-points for APTs (Advanced Persistent Threat) in a huge number of companies that comply with with BYOD (Bring Your Own Device) policy for cost reduction and personal convenience. 

While existing tools like PCAPdroid and Ant-Monitor provide traffic analysis and monitoring capabilities, they often lack integration with real-time threat recognition. This project exhibits the design and implementation of an Android-based threat detection application that leverages the android VPNService API to capture and intercept network/internet traffic. This comes alongside the functionality to map associated packets to originating device applications. This thesis project incorporates AbuseIPDB. A well-known platform dedicated to helping users and administrators combat the spread of hackers, spammers, and abusive activity on the internet. This incorporation is to assess the maliciousness of destination IP addresses in the outgoing internet packets, notifying the user of the corresponding potential risk(s) that the application can introduce.

This application is developed as a complementary addition to PCAPdroid that lacks live threat detection and analysis of network/internet traffic. It utilizes the passive packet-capture capabilities of PCAPdroid and employs AbuseIPDB capabilities to bridge the gap between packet capture and live threat analysis combined with the latest state-of-the-art user interface approaches. 

This application receives the outgoing IP address, application UIDs to extract the app-specific information alongside other useful data in the form of a PCAPNG file via a local TCP Server from PCAPdroid and subsequently transmits and inquiry to AbuseIPDB to evaluate the maliciousness of outbound traffic.

Threat Detector illustrates an ability to identify suspicious connections with minimal performance and storage overhead, highlighting it as a potent and practical tool to enhance mobile security, privacy and user awareness.
