% !TeX encoding = UTF-8
% !TeX spellcheck = en_GB
% !TeX root = ../thesis.tex

\chapter{Conclusion}
\label{chapter:conclusions}

This thesis presented the design, implementation, and evaluation of Threat Detector; an Android-based extension application developed to enhance mobile network security by identifying potentially malicious network connections. The system operates by a thorough integration with PCAPdroid, a network monitoring tool that leverages Android's VpnService API to capture, analyse, and export network traffic. Rather than relying on an implementation of VpnService, Threat Detector receives traffic data in the form of PCAP files which are then analysed to determine the maliciousness of the connections established by various applications on the Android device. By implementing external threat intelligence through AbuseIPDB, Threat Detector aims to bridge the gap between raw packet capture and real-time threat analysis on Android devices.

\section{Summary of the Achieved Objectives}
The main goal of this thesis project was to develop a solution for identifying malicious IP addresses from mobile network traffic in an automated and user-friendly manner. This required the integration of multiple components including network traffic capture, packet analysis, and reputation lookups. The approach was implemented to remain lightweight and non-intrusive (passive packet capture) that allows users to maintain normal device functionality without considerable interference in network performance.

The first achievement of this project was establishing a reliable communication channel for data exchange between Threat Detector and PCAPdroid. PCAPdroid was configured to export the captured data through its TCP exporter functionality in the form of a PCAP file that has a certain structure which was used to extract the captured information. This has enabled Threat Detector to process packets in real-time on their arrival that can theoretically lead to less overhead in comparison to offline analysis. This setup ensured that network data could be collected in a structured manner through a seamless communication using a client-server implementation.

Another milestone for this project to achieve was to employ a functional analysis logic that is responsible for reading captured packets, extracting destination IP addresses, and performing reputation checks using AbuseIPDB API. By querying AbuseIPDB database, Threat Detector could determine whether an IP address has been previously reported for malicious activities, including spam, DDoS (Distributed Denial of Service), remote access, or other network attacks. In case of a connection to a malicious IP address, the results were shown in an easily interpretable format, enabling the user or the administrator to evaluate their network's safety.

A further accomplishment achieved by Threat Detector was the implementation of IP-to-application mapping mechanism between the captured packets and their originating application. This was implemented through UID-to-package resolution. It is an API provided by Android that enabled Threat Detector to associate IP addresses with their originating Android application. This implementation transformed Threat Detector from a purely network-oriented tool into a an application aware analysis tool, allowing the user to identify whether their applications are communicating with malicious back-end sources.

As a result, the combination of aforementioned features form a comprehensive framework, using which the users can monitor, analyse, and interpret the network traffic of their Android devices in a practical and user-friendly way. Threat Detector demonstrated how combining open-source traffic capturing tools such as PCAPdroid and external threat intelligence services such as AbuseIPDB can significantly improve application transparency and awareness in mobile network security.


\section{Reflection on System Performance}
The assessment of Threat Detector in combination with PCAPdroid has highlighted both its strengths and limitations. From a functional perspective the communication between the applications proved to be reliable and efficient. The exported traffic data from PCAPdroid was correctly parsed and the extraction of destination IP addresses were consistent. The integration of AbuseIPDB also functioned as intended; the system has successfully received the IP reputation data and depicted the results in an easy-to-interpret form.

Performance testing, as discussed in the previous chapter, has shown that Threat Detector was capable of receiving and analysing files of small to moderate size without excessive delay or drastic resource consumption. However, the reliance on PCAPdroid as a stand-alone application to transfer PCAP files introduced some latency between capture and analysis to some degree that could be circumvented using a built-in VpnService implementation.

This project also showed that the accuracy of threat detection significantly relies on the quality and the freshness of the data provided by AbuseIPDB. Even though it is widely used and frequently updated, it cannot guarantee a complete coverage of malicious IP addresses specifically for new or short-lived threats. Nonetheless, the employment of AbuseIPDB in this project offered a practical solution that demonstrated the feasibility of integrating threat intelligence into mobile network monitoring systems.

\section{Encountered Challenges}
Developing Threat Detector presented several technical challenges. One of the most intricate challenges was parsing and interpreting the PCAP file received from PCAPdroid. Since PCAPdroid extensions has been activated to include more data in the PCAP file, it has complicated the process of data extraction. Although the process is thoroughly explained as a walk-through on PCAPdroid's official website, finding the \texttt{magic trailer} number that indicates the beginning of the PCAP file content was the most significant challenge faced throughout the development of Threat Detector.


Another challenge was to correctly associate the UID values with the application names. Since the UID values should be first mapped to the application package and then the name is extracted from the package data, the generated traffic by background services can introduce some problems. The reason behind it is that the background services have an associated package but not any data regarding application name and icon. This has been partially handled by ignoring the private destination IP addresses, since the background service traffic is directed to a local gateway before leaving the network.

Finally, the API rate limit for AbuseIPDB queries introduced a practical challenge. As the system totally relies on external lookups for every single IP address, maintaining efficiency while making the most out of free API calls required implementation of local caching mechanisms and hardcoding the most prominent IP addresses (FAANG).

\section{Contributions to the Field}
The development of Threat Detector demonstrated in this thesis contributes to the growing filed of mobile network forensics and threat intelligence integration. Most of the existing Android security tools work as either a file-based antivirus or network monitors that lack the integration of threat intelligence databases. Threat Detector combines these features by providing a lightweight and modular framework that connects network activities to threat intelligence and application context. This approach has provided a foundation for future work to integrate similar approaches in the field of mobile security.

This project also demonstrates the practicality of using existing open-source solutions like PCAPdroid and extending their functionality rather than building a fully-fledged solutions from scratch.

\section{Limitations}
In spite of the accomplishments, Threat Detector has some certain limitations that are worth to mention. First, the system dependence on PCAP files meaning that it does not perform any live packet interception or active IP/application blocking. As a result, the application functions primarily as a diagnostic and monitoring tool rather than a real-time protection system. This was an intentional trade-off to avoid complications regarding the establishment and maintenance of a persistent VPN connection.

Second, the reliance on AbuseIPDB for threat intelligence introduces another dependency on an external source that might not be always available or may impose usage restrictions. Additionally, reputation-based systems have a degree of reliability; they can produce false positive or false negatives depending on the quality of the community reports. Expanding this system to integrate various threat intelligence sources can increase the reliability of the queried reputation checks.

Finally, the system's evaluation was conducted under a controlled condition and with a small range of Android devices. A broader study involving different brands and larger datasets along with more applications participating in the network traffic generation can help evaluating the system's performance and scalability.


\section{Final Remarks}
In conclusion, this thesis demonstrated the feasibility and effectiveness of combining mobile traffic monitoring systems with external threat intelligence services to enhance the previous approaches towards mobile network security. Through the integration of PCAPdroid and AbuseIPDB, Threat Detector provides useful insights about the application-level network behaviour and potential security risks they bring along. Although the system is primarily analytical and passive, its modular design lays the foundation for further development and eases the integration of additional features. Threat Detector Android application is a prototype that showed how a combination of modular, open-source, and intelligence-oriented solutions can improve transparency and raise user's awareness in the context of mobile security.

