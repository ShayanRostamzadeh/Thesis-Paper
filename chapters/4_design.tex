% !TeX encoding = UTF-8
% !TeX spellcheck = en_GB
% !TeX root = ../thesis.tex

\chapter{Architecture/System Design}
\label{chapter:design}

\section{Overview}
This chapter represents the design architecture of the proposed mobile network monitoring system, Threat Detector. This application aims to detect the potentially malicious network traffic on Android devices with a passive packet sniffing approach. It is developed as an enhancement and an extension for the open-source applicaion PCAPdroid, integrating additional modules for IP reputation checking and application-to-ip mapping, while using the state-of-the-art user interface and user experience implementations to meet the design expectations of a modern Android application.

As covered further in this chapter, Overview elaborates on system components, depicting a modular approach that is employed to develop Threat Detector. It is then followed by some illustrations, explaining the application's function though the use of a flow diagram. Finally, the reasons behind those decisions are summarized.

Application's architecture is designed based on three main objectives:

\begin{itemize}
	\item \textbf{Transparency:} Intercepting and logging of network traffic without altering or blocking the normal traffic flow.
	\item \textbf{Association:} Accurately associating each network connection with its originating application.
	\item \textbf{Threat Detection:} Assessing the reputation of the destination IP address with the assistance of external IP intelligence sources such as AbuseIPDB.
\end{itemize}

The overall structure of Threat Detector follows a modular design which ensures separation between packet capture, user interface, and data processing.

As shown in \cref{fig:system_architecture_overview}, the first layer is handled by the Android operating system which provides access to the applications' network traffic. Second layer is implemented by PCAPdroid that forwards all of the sniffed packets to a local gateway for centralized analysis. As the final layer that PCAPdroid is responsible for, data assembly layer builds the PCAP/PCAPNG files and transmits them to a local server using TCP. Subsequently, Threat Detector analyses the received data, queries AbuseIPDB for IP reputation, and notifies the user.


\begin{figure}[H]
	\centering
	\begin{tikzpicture}[
		node distance=1.2cm,
		every node/.style={rectangle, rounded corners, draw=black!70, thick, align=center, font=\footnotesize, minimum width=8cm, minimum height=1cm, fill=gray!5},
		arrow/.style={->, thick}
		]
		
		% Layers
		\node (apps) [minimum width=5cm, fill=orange!20] {Android Applications\\(WhatsApp, Google Chrome, Instagram, etc.)};
		
		\node (vpn) [minimum width=5cm, above=of apps, fill=blue!5] {Packet Capture Layer\\\textit{VPNService-based tunneling Virtual Interface}\\Packet Sniffer/Forwarder (PCAPdroid Core)};
		
		\node (assembly) [minimum width=5cm, above=of vpn, fill=green!5] {Data Assembly Layer \\PCAPdroid PCAP/PCAPNG Assembly \\UID Resolver};
		
		\node (threat) [right=4cm of assembly, fill=red!5, minimum width=5cm] {Threat Detector App \\Data Processing Layer \& App Mapper \\PCAP Receiver \& Analysis \\AbuseIPDB Reputation Checker\\Local Information Storage};
		
		\node (ui) [above=of threat, fill=yellow!10, minimum width=5cm] {User Interface Layer\\Maliciousness Notification \\IP Reputation Score \\Report Details};
		
		
		\node (tcpnode) [right=1cm of assembly, draw, align=center, font=\scriptsize, minimum width=1.8cm, minimum height=0.8cm] {Local TCP Server};
		
		% Connections
		\draw[arrow] (apps) -- (vpn);
		\draw[arrow] (vpn) -- (assembly);
		\draw[arrow] (threat) -- (ui);
		
		
		\draw[<->, dashed, thick] (assembly.east) -- (tcpnode.west);
		\draw[<->, dashed, thick] (tcpnode.east) -- (threat.west);
	
			
	\end{tikzpicture}
	\caption{Modular System Architecture of Threat Detector, Own Creation}
	\label{fig:system_architecture_overview}
\end{figure}



\section{System Components}
Threat Detector as a systems is composed of several independent stand-alone components responsible for a seamless functionality that follow the "Separation of Concerns" model for ease of debugging and further development. 


\subsection{VPN-based packet Capture Module}
This module establishes a virtual private network interface (through utilization of PCAPdroid) that captures the incoming and outgoing packets from user's applications without requiring root access on the Android device. Raw captured packets are then immediately forwarded to the next module for further analysis while keeping the communications between endpoints alive.

The functions of this module are as follows:
\begin{itemize}
	\item Initialization and maintenance of a virtual interface (tunnel interface).
	\item Capturing raw packets from the device's network traffic while preventing the modification of the packets.
	\item Forwarding the packets to the local processing engine of PCAPdroid.
\end{itemize}

%\clearpage

\subsection{UID Resolver and Application Mapper}
This module takes the responsibility of correlating raw captured packets with the originating Android applications. Each packet commuting through the network has a retrievable UID that can be extracted using system calls. As the last step, this module employs Android's \emph{PackageManager}\footnote{\url{https://developer.Android.com/reference/Android/content/pm/PackageManager}} to map the UIDs to human-readable application package names.


\subsection{AbuseIPDB Reputation Checker}
This component integrates the use of external threat intelligence sources such as AbuseIPDB. For each of the unique IP addresses detected, it queries a reputation check determining whether the IP address has been reported as malicious/suspicious.

Functions of this module are as follows:
\begin{itemize}
	\item Sending HTTP requests to AbuseIPDB using its API querying IP's maliciousness.
	\item Parsing the JSON response which contains IP's confidence score and report details.
	\item Caching results for higher performance.
	\item Communicating reputation data with the UI layer for user depiction.
\end{itemize}

This module comes with some the following design considerations:
\begin{itemize}
	\item Efficient querying avoiding redundant API calls.
	\item Asynchronous execution of reputation checks to prevent UI blocking using an additional thread.
	\item Respecting API rates and response interpretation.
\end{itemize}


\subsection{Inter-Application Communication}
This module enables an enhanced architecture to include a local TCP servers for the communication between PCAPdroid and Threat Detector to take place. The captured data is then transferred between the two in a form of PCAP/PCAPNG file.

Functions of this module are as follows:
\begin{itemize}
	\item Creation and establishment of a TCP socket server within Threat Detector.
	\item Transmission of captured packets from PCAPdroid to Threat Detector.
	\item Enabling Threat Detector to parse, analyse, and visualize data for user.
\end{itemize}

The aforementioned design ensure modularity and scalability. It allows Threat Detector to evolve independently while receiving live traffic information from PCAPdroid.


\subsection{Data Storage Agent and User Interface Layer}
This module is not only responsible to cache the most queried IP addresses in the application's RAM, but also stores the API key and user's data in \emph{SharedPreferences}\footnote{\url{https://developer.Android.com/training/data-storage/shared-preferences}}.


\subsection*{User Interface}
Threat Detector's user interface presents network and threat information in an organized intuitive format. It allows the user to view ongoing connections associated with the applications, inspect detailed IP-related information including confidence score and a report explaining the reason behind the IP being flagged as malicious.

%Design goals:
%\begin{itemize}
%	\item Simple, responsive, and modern layout.
%	\item Clear visualization and notification off malicious indicators.
%	\item Use of newly introduced approaches of UI implementation such as \href{https://developer.Android.com/compose}{\emph{Jetpack Compose}} and \href{https://developer.Android.com/develop/ui/compose/lists}{Lazy Column} for a performance.
%\end{itemize}



\section{Data Flow and Module Interaction}
The data flow (\cref{fig:Threat_Detector_Flow_Diagram}) follows a clear path from packet capture in PCAPdroid to threat assessment in Threat Detector as depicted below.

\begin{figure}[H]
	\centering
	\includegraphics[width = 0.55\textwidth]{Flow_Diagram.png}
	\caption{Threat Detector Flow Diagram, Own Creation}
	\label{fig:Threat_Detector_Flow_Diagram}
\end{figure}

\clearpage

The implementation this project follows to fulfill threat assessment, can be explained by steps below:
\begin{itemize}
	\item \textbf{VpnService Capture:} Traffic from all of the applications are accumulated and redirected to the virtual interface; in this case default gateway.
	\item \textbf{Packet Capture and Forwarding:} Packets are handed over to PCAPdroid processor for UID and destination IP address extraction, followed by mapping to the originating application.
	\item \textbf{Reputation Check:} Destination IP addresses are queried against AbuseIPDB.
	\item \textbf{Score Caching:} Destination IP address and its associated confidence score are stored in a map for faster future access and minimization of API calls.
	\item \textbf{User Notification and Info Depiction:} The confidence score, IP report, and its originating application are depicted to the user. Afterwards, the score is checked against a minimum user-set value that consequently notifies the user if it is higher than the set value.
\end{itemize} 


\section{Design Decisions}
Various design choices were made throughout this project to achieve performance and modularity alongside scalability. \cref{tab:design_decisions} shows these justifications.

\begin{table}[H]
	\centering
	\caption{Design Decisions and Their Justifications, Own Creation}
	\label{tab:design_decisions}
	\begin{tabular}{p{4cm} p{10cm}}
		\toprule
		\textbf{Design Decision} & \textbf{Design Justification} \\
		\midrule
		Use of \texttt{VPNService} instead of root access &
		Provides a passive method for traffic interception without requiring elevated privileges, which ensures compatibility with non-rooted devices while preserving user privacy. \\
		
		\midrule
		
		Extension of PCAPdroid framework &
		Leverages an existing open-source foundation for packet capture and logging, allowing focus to be only on threat intelligence rather than reimplementation of capture logic. \\
		
		\midrule
		
		Integration of a local TCP server &
		Ensures modular communication through a lightweight socket connection, enabling independent development and scaling of the Threat Detector application as a stand-alone solution. \\
		
		\midrule
		
		Asynchronous API calls for AbuseIPDB queries &
		Prevents blocking of the user interface thread and maintains responsiveness during IP reputation lookups. \\
		
		\midrule
		
		Caching of Report results &
		Reduces redundant requests, achieving optimization and reduction of API usage and faster access for lookups. \\
		
		\midrule
		
		Modular and layered architecture &
		Improves maintainability, scalability, and fault isolation, allowing each component (capture, mapping, reputation check, and UI) to function independently. \\
		
		\midrule
		
		User Notification &
		Keeps the user or administrator informed about potential malicious activities with the use of real-time notification system that includes application name and its IP address. \\
		\bottomrule
	\end{tabular}
\end{table}

\clearpage


\section{Summary}
The design architecture presented in this chapter provides a reliable foundation for packet capture, application network monitoring, and threat detection on Android devices. 

Threat Detector's modular design facilitates easy updates and further development to implement additional APIs and features. Moreover, the use of VpnService ensures compatibility with various Android versions while preventing the need for root access when it comes to packet interception and application UID extraction.









